\documentclass[12pt,fleqn]{article}

%\usepackage{graphicx}
\usepackage{url}
%\usepackage{verbatim}
\usepackage{color}
\usepackage{amsmath}
\usepackage{amssymb}
\usepackage{amsfonts}
\usepackage{enumitem}
%\usepackage{multicol}
%\usepackage[tight]{subfigure}\subfigtopskip=0pt

\numberwithin{equation}{section}
%\numberwithin{figure}{section}

\voffset=-1in
\hoffset=-1in
\topmargin=1.00in
\oddsidemargin=1.25in
\evensidemargin=1.25in
\textwidth=6.5in
\textheight=9.0in
%% comment out the next two lines to save space
\parskip=12pt
\parindent=0pt
\footskip=60pt
\marginparsep=1cm
\marginparwidth=1cm
\headheight=0pt
\headsep=0pt

\pagestyle{empty}

% This is supposed to select the Times New Roman font.
\usepackage{pslatex}

\setcounter{secnumdepth}{3}

%\makeatletter
%\def\section{\@startsection
% {section}{1}{0mm}%   % name, level, indent
% {-\baselineskip}%    % beforeskip
% {0.5\baselineskip}%  % afterskip
% {\bf}}%              %style
%\def\subsection{\@startsection
% {subsection}{2}{32pt}%% name, level, indent
% {-0pt}%    % beforeskip
% {-\baselineskip}%    % afterskip
% {\bf}}%              %style
%\def\subsubsection{\@startsection
% {subsubsection}{3}{32pt}%% name, level, indent
% {-0pt}%    % beforeskip
% {-\baselineskip}%    % afterskip
% {\itshape}}%              %style
%\makeatother

\begin{document}

\noindent
{\Large \bf Collisionless fast reconnection with an isotropic two-fluid model}

The code in this directory solves 
the Geospace Environmental Modeling Magnetic Reconnection
Challenge Problem (alias ``GEM reconnection problem'') \cite{article:Birn01}
using an isotropic two-fluid collisionless model.

%\section{Background and context}

\section{System of equations: isotropic two-fluid plasma}

The code in this directory solves
the collisionless five-moment two-fluid model,
which models each species as an ideal gas
coupled to Maxwell's equations.

The nondimensionalized system of equations that
we solve has the same form as the dimensional equations:
\def\wh{\widehat}
\def\pressure{p}
\def\E{\mathbf{E}}
\def\B{\mathbf{B}}
\def\J{\mathbf{J}}
\def\Div{\nabla\cdot}
\def\curl{\nabla\times}
\def\debyeLength{\lambda_D}
%\def\gyroRadius{r_g}
\def\gyrofrequency{\omega_g}
\def\mdens{\rho}
\def\u{\mathbf{u}}
\def\gasenergy{\mathcal{E}}
\def\idtens{\mathbb{I}}  % {\underline{\underline{\delta}}}
\def\qdens{\sigma}
\def\cc{{\chi}}%{{c_c}}
\def\diminished{\color{cyan}}
\def\emphasized{\color{blue}}
\begin{gather*}
 \partial_{\wh t}
    \begin{bmatrix}
      \wh\mdens_i \\
      \wh\rho_i\wh\u_i \\
      \wh\gasenergy_i \\
      \wh\mdens_e \\
      \wh\rho_e\wh\u_e \\
      \wh\gasenergy_e \\
    \end{bmatrix}
   +
   \wh \Div
    \begin{bmatrix}
      \wh\rho_i\wh\u_i \\
      \wh\rho_i\wh\u_i\wh\u_i + \wh\pressure_i \, \idtens \\
      \wh\u_i\bigl(\wh\gasenergy_i+\wh\pressure_i\bigr)\\
      \wh\rho_e\wh\u_e \\
      \wh\rho_e\wh\u_e\wh\u_e + \wh\pressure_e \, \idtens \\
      \wh\u_e\bigl(\wh\gasenergy_e+\wh\pressure_e\bigr)\\
    \end{bmatrix}
 = %\wh\gyrofrequency %\frac{1}{\wh\gyroRadius}
    \begin{bmatrix}
      0 \\
      \wh\qdens_i(\wh\E+\wh\u_i\times\wh\B)\\
      \wh\qdens_i\wh\u_i\cdot\wh\E \\
      0 \\
      \wh\qdens_e(\wh\E+\wh\u_e\times\wh\B)\\
      \wh\qdens_e\wh\u_e\cdot\wh\E \\
    \end{bmatrix},
 \\
 \partial_{\wh t}
    \begin{bmatrix}
      \wh c\wh\B \\
      \wh\E \\
    \end{bmatrix}
 + \wh c
    \begin{bmatrix}
      \wh\curl\wh\E{\diminished +\cc\nabla\psi} \\
      -\wh\curl(\wh c\wh\B){\diminished + \cc\nabla(\wh c\phi)} \\
    \end{bmatrix}
 = \begin{bmatrix}
      0 \\
      -\wh\J/{\wh\epsilon} \\
    \end{bmatrix}, \hbox{ and }
 %\\
 {\diminished
 \partial_{\wh t}
    \begin{bmatrix}
      {\diminished\psi} \\
      {\diminished \wh c\phi} \\
    \end{bmatrix}
 + \wh c
 }
    \begin{bmatrix}
       {\diminished \cc{\emphasized \wh \Div(\wh c\wh\B)}} \\
       {\diminished \cc{\emphasized \wh\Div\wh\E} } \\
    \end{bmatrix}
 = \begin{bmatrix}
      {\emphasized 0} \\
      {\diminished \wh c\cc{\emphasized \wh\qdens/\wh\epsilon}} \\
    \end{bmatrix}.
\end{gather*}
Hats denote quantities that are rescaled or redefined
in the nondimensionalization.
The dimensional equations have the same appearance except that
there are no hats.
The quantities are defined as follows.
Let $s\in\{i,e\}$ denote species index 
($i$ for ion, $e$ for electron). Then
$\rho_s$ is density,
$\u_s$ is fluid velocity,
$\gasenergy_s$ is gas-dynamic energy,
$p_s$ is gas-dynamic scalar pressure,
$\idtens$ is the identity tensor,
$\E$ is electric field,
$\B$ is magnetic field,
$\sigma_s=\frac{q_s}{m_s}\rho_s$ is charge density,
$q_s$ and $m_s$ are particle mass and charge,
%$q_i=1$, $q_e=-1$, $m_i=1$, $m_e=\frac{m_e}{m_i}$,
and
$\J=\J_i+\J_e=\qdens_i\u_i+\qdens_e\u_e$ is current.
The constitutive relation for the
scalar pressure is given by
$\gasenergy=(\alpha/2)p+(1/2)\mdens u^2$,
where $\alpha$, the number of degrees of freedom
per particle, equals 3.
For the parameters,
$c$ is the speed of light and
$\epsilon$ is the electric permittivity.
The quantities $\diminished\psi$ and $\diminished\phi$ are zero
for physical solutions; they are included
in the system to maintain
the divergence constraints.  We choose
$\diminished\chi$ to ensure that divergence error
propagates at least as fast as the fastest physical
waves in our system, in our case light waves.
So $\diminished\chi$
must be at least $1$; we have chosen $\diminished\chi=1.05$.

In our nondimensionalization we take a typical
angular gyroperiod as our time scale and
a typical Alfv\'en speed as our velocity scale;
then $1/\wh\epsilon=\wh c^2$.

Specifically, to nondimensionalize
we choose typical magnetic field strength $B_0$,
particle density $n_0$, particle mass $m_0$
(which we take to be the total mass of an ion plus an electron),
charge $q_0$ (which we take to be the charge $e$ on a proton),
and choose time scale
$t_0=\frac{m_0}{q_0 B_0}$, the angular gyroperiod
of a typical particle, and typical velocity
$v_A = \frac{\mu_0 m_0 n_0}{B_0^2}$, a typical Alfv\'en speed.
Then
$\frac{1}{\wh\epsilon} = \frac{t_0 n_0 e}{B_0 \epsilon_0}
   = \frac{\mu_0 m_0 n_0}{B_0^2} c^2
   = \big(\frac{c}{v_A}\big)^2=\wh c^2$,
where $\wh c$ denotes nondimensionalized light speed.

Note that the ion pressure obeys the evolution equation
$(d^{\u_i}_t)\pressure_i+\gamma(\nabla\cdot\u_i)\pressure_i=0$,
where $(d^{\u_i}_t):=d_t+\u_i\cdot\nabla$
is the convective derivative defined by $\u_i$,
and $\gamma=\frac{\alpha+2}{\alpha}$.  Likewise for the
electron pressure.  These pressures should remain positive.

\section{Boundary and initial conditions; GEM reconnection challenge problem.}

The Geospace Environmental Modeling magnetic reconnection
challenge problem is a standard test problem to test whether
a model admits fast reconnection.

\subsection{Computational domain.}
The computational domain is the rectangular domain
$[-L_x/2,L_x/2]\times[-L_y/2,L_y/2]$,
where $L_x=8\pi$ and $L_y=4\pi$.
The problem is symmetric under reflection across
either the horizontal or vertical axis.

\subsection{Boundary conditions.}
The domain is periodic in the $x$-axis.
The boundaries perpendicular to the $y$-axis
are thermally insulating conducting wall boundaries.  
A conducting wall boundary is a solid wall boundary
(with slip boundary conditions in the case of ideal plasma)
for the fluid variables, and the electric field at
the boundary has no component parallel to the boundary.
Assuming the ideal MHD Ohm's law, this implies that
at the conducting boundary the magnetic field must be parallel to 
the boundary.  So at the conducting wall boundaries
\begin{align*}
 \\ \partial_y u_{x,s} &= 0,
 &  \partial_y B_x &= 0,
 &  E_x &= 0,
 & \partial_y \rho_s &= 0,
 %
 %
 \\ u_{y,s} &= 0,
 &  B_y &= 0,
 &  \partial_y E_y &= 0,
 & \partial_y p_s &= 0,
 %
 \\ \partial_y u_{z,s} &= 0,
 &  \partial_y B_z &= 0,
 &  E_z &= 0,
 & \partial_y \gasenergy_s &= 0.
\end{align*}

\subsection{Model Parameters.}
In the GEM problem the model parameters are
\begin{align*}
   m_i/m_e &= 25,
 & \mu_0 &= 1.
\end{align*}

Following the two-fluid simulations of
\cite{article:Loverich05} and \cite{article:Hakim06},
we choose a speed of light
\begin{align*}
  c = 10 B_0,
\end{align*}
i.e. 10 times the Alfv\`en speed.

For our nondimensionalization parameters we assumed that
the typical gyrofrequency and Alfv\'en speed are unity.

\subsection{Initial conditions.}
The initial conditions are a perturbed Harris sheet equilibrium.
The unperturbed equilibrium is given by
\def\e{\mathbf{e}}
\def\B{\mathbf{B}}
\def\E{\mathbf{E}}
\def\J{\mathbf{J}}
\def\sech{\,\mathrm{sech}}
\begin{align*}
    \B(y) & =B_0\tanh(y/\lambda)\e_x,
  & p(y) &= \frac{B_0^2}{2 n_0} n(y),
 \\ n_i(y) &= n_e(y)
            = n_0(1/5+\sech^2(y/\lambda)),
  & p_e(y) &= p(y)/6,
 \\ \E & =0,
  & p_i(y) &= 5p(y)/6.
\end{align*}

On top of this the magnetic field is perturbed by
\begin{align*}
   \delta\B&=-\e_z\times\nabla(\psi), \hbox{ where}
\\ \psi(x,y)&=\psi_0 \cos(2\pi x/L_x) \cos(\pi y/L_y).
\end{align*}
In the GEM problem the initial condition constants are
\begin{align*}
    \lambda&=0.5,
  & B_0&=1,
 \\ n_0&=1,
  & \psi_0&=B_0/10.
\end{align*}

We remark that \cite{article:Hakim06} instead sets $B_0=0.1$
(so that the speed of light will be unity when it
is 10 times the Alfv\'en speed).
His solution maps onto the GEM variables via
\def\GEM{\mathrm{GEM}}
\def\u{\mathbf{u}}
\begin{align*}
   \B&=.1\B_\GEM,
 &  t&=10 t_\GEM,
 &  p&=.01 p_\GEM,
 & (\rho_s)=(\rho_s)_\GEM,
\\ \E&=.01\E_\GEM,
 & \u&=.1\u_\GEM,
 & \gasenergy&=.01 \gasenergy_\GEM,
\end{align*}
and so forth.

In his two-fluid simulations,
\cite{article:Hakim06}, assumes that the
initial current is carried only by the electrons
and neglects the portion of the current caused by the
initial perturbation:
\begin{gather*}
   \J_e = \curl \B = -\frac{B_0}{\lambda}\sech^2(y/\lambda) \e_z.
\end{gather*}
In the case of a mass ratio near unity and large perturbation,
we prefer a more exact calculation of the initial velocities.

We calculate the precise initial current:
\begin{align*}
     \J =& \curl (\B + \delta\B).
%\\  \B  =& B_0\tanh(y/\lambda)\e_x.
\\ \curl\B =& -\frac{B_0}{\lambda}\sech^2(y/\lambda) \e_z.
%\\ \psi(x,y)=&\psi_0 \cos(2\pi x/L_x) \cos(\pi y/L_y).
\\ \delta\B=&-\e_z\times\nabla \psi
   =  \begin{bmatrix}
        \partial_y \psi \\
       -\partial_x \psi \\
        0
      \end{bmatrix}
   =  \begin{bmatrix}
        -\psi_0(\pi/L_y) \cos(2\pi x/L_x) \sin(\pi y/L_y)  \\
         \psi_0(2\pi/L_x) \sin(2\pi x/L_x) \cos(\pi y/L_y) \\
        0
      \end{bmatrix}
\\ \curl\delta\B =& -(\nabla^2\psi)\e_z
       =  \psi_0 ((2\pi/L_x)^2+(\pi/L_y)^2) \cos(2\pi x/L_x) \cos(\pi y/L_y)\e_z.
\\ \J =& (-\frac{B_0}{\lambda}\sech^2(y/\lambda) - \nabla^2 \psi) \e_z.
\end{align*}
From the current we calculate the initial species currents and
velocities.  Assuming zero initial net momentum gives the system
\begin{align*}
  \J =& \J_i + \J_e, \\
   0 =& m_i\J_i - m_e\J_e,
\end{align*}
whose solution is
\begin{align*}
  \J_i =& \frac{m_e}{m_i+m_e} \J, \\
  \J_e =& \frac{m_i}{m_i+m_e} \J.
\end{align*}
So the initial momentum of each species is:
\begin{align*}
  \rho_i\u_i =& m_i \J_i/e = m \J/e, \\
  \rho_e\u_e =& -m_e \J_i/e = -m \J/e,
\end{align*}
where
$m = \frac{m_i m_e}{m_i+m_e}$ is the \emph{reduced mass}.

\appendix

\section{Nondimensionalization}

A slightly more generic nondimensionalized system of equations than
ours has essentially the same form as the dimensional equations:
\def\wh{\widehat}
\def\pressure{p}
\def\E{\mathbf{E}}
\def\B{\mathbf{B}}
\def\J{\mathbf{J}}
\def\Div{\nabla\cdot}
\def\curl{\nabla\times}
\def\debyeLength{\lambda_D}
%\def\gyroRadius{r_g}
\def\gyrofrequency{\omega_g}
\def\mdens{\rho}
\def\u{\mathbf{u}}
\def\gasenergy{\mathcal{E}}
\def\idtens{\mathbb{I}}  % {\underline{\underline{\delta}}}
\def\qdens{\sigma}
\begin{gather*}
 \partial_{\wh t}
    \begin{bmatrix}
      \wh\mdens_i \\
      \wh\mdens_e \\
      \wh\rho_i\wh\u_i \\
      \wh\rho_e\wh\u_e \\
      \wh\gasenergy_i \\
      \wh\gasenergy_e \\
    \end{bmatrix}
   +
   \wh \Div
    \begin{bmatrix}
      \wh\rho_i\wh\u_i \\
      \wh\rho_e\wh\u_e \\
      \wh\rho_i\wh\u_i\wh\u_i + \wh\pressure_i \, \idtens \\
      \wh\rho_e\wh\u_e\wh\u_e + \wh\pressure_e \, \idtens \\
      \wh\u_i\bigl(\wh\gasenergy_i+\wh\pressure_i\bigr)\\
      \wh\u_e\bigl(\wh\gasenergy_e+\wh\pressure_e\bigr)\\
    \end{bmatrix}
 = \wh\gyrofrequency %\frac{1}{\wh\gyroRadius}
    \begin{bmatrix}
      0 \\
      0 \\
      \wh\qdens_i(\wh\E+\wh\u_i\times\wh\B)\\
      %-\frac{m_i}{m_e}
      \wh\qdens_e(\wh\E+\wh\u_e\times\wh\B)\\
      \wh\qdens_i\wh\u_i\cdot\wh\E \\
      %-\frac{m_i}{m_e}
      \wh\qdens_e\wh\u_e\cdot\wh\E \\
    \end{bmatrix},
 \\
 \partial_{\wh t}
    \begin{bmatrix}
      \wh c\wh\B \\
      \wh\E
    \end{bmatrix}
 + \wh c \wh \curl
    \begin{bmatrix}
      \wh\E \\
      -\wh c\wh\B
    \end{bmatrix}
 = \begin{bmatrix}
      0 \\
      -\wh\J/{\wh\epsilon}
    \end{bmatrix}, \hbox{ and }
  \wh\nabla\cdot
   \begin{bmatrix}
     \wh c\wh\B \\
     \wh\E
   \end{bmatrix}
 = \begin{bmatrix}
     0 \\
     \wh\qdens/\wh\epsilon
   \end{bmatrix}.
\end{gather*}

To nondimensionalize we choose a typical particle
density $n_0$, magnetic field strength $B_0$,
time scale $t_0$, speed $u_0$, and particle mass $m_0$,
and charge $q_0$.
We then make the appropriate substitutions, e.g., 
\begin{align*}
  \rho &=\rho_0\wh\rho,
 &\u   &=u_0\wh\u,
 &\gasenergy &= \gasenergy_0\wh\gasenergy,
 &\B &= B_0\wh\B,
 &\J &= J_0\wh\J,
\\
  \sigma &= \sigma_0\wh\sigma,
 &c    &=u_0\wh c,
 &\pressure &= \pressure_0\wh\pressure,
 &\E &= E_0\wh\E,
\end{align*}
where
\begin{align*}
  \rho_0 &= m_0 n_0,
 &\sigma_0 &= q_0 n_0,
 &\gasenergy_0 &= \rho_0 u_0^2 = \pressure_0,
 &E_0 &= u_0 B_0,
 &J_0 &= \sigma_0 n_0 u_0.
\end{align*}
We choose $m_0$ to be the charge of a protium
(hydrogen) atom and $q_0=e$, the charge of a proton.

For the three nondimensional parameters:
$\gyrofrequency:=\frac{q_0 B_0}{m_0}$
is a typical gyrofrequency and
$\wh\gyrofrequency=t_0\gyrofrequency$.
In the GEM problem the time
scale $t_0$ is chosen so that $\wh\gyrofrequency=1$.

$\frac{1}{\wh\epsilon} = \frac{t_0 n_0 e}{B_0 \epsilon_0}
   = t_0 \frac{e B_0}{m_0} \frac{\mu_0 m_0 n_0}{B_0^2} c^2
   = (t_0 w_g) \big(\frac{c}{v_A}\big)^2$
serves as a pseudo-permittivity
($\epsilon_0$ and $\mu_0$ are the permittivity 
and permeability of free space, $c$ is the
speed of light, and $v_A$ is a typical
Alfv\'en speed).  In the GEM problem the typical
velocity $u_0$ is chosen to be the Alfv\'en speed $v_A$
so that $\frac{1}{\wh\epsilon} =\wh c^2$.

\begin{thebibliography}{99}

\bibitem[Birn01]{article:Birn01}
    Birn, J., et al.,
    \newblock {\em Geospace Environmental Modeling (GEM)
      magnetic reconnection challenge},
    J. Geophys. Res., 106(A3), 3715--3720, 2001.

\bibitem[Dedner02]{article:Dedner02}
    A. Dedner, F. Kemm, D. Kr\:oner, C.-D. Munz, 
    T. Schnitzer, and M. Wesenberg,
    {\em Hyperbolic divergence cleaning 
    for the MHD equations},
    Journal of Computational Physics {\bf 175}, 645--673 (2002).

\bibitem[Hakim06]{article:Hakim06}
     A. Hakim, J. Loverich, U. Shumlak,
     \newblock \emph{A high resolution wave propagation scheme for
       ideal two-fluid plasma equations},
     \newblock Journal of Computational Physics, 219 (2006) 418-442.

\bibitem[Loverich05]{article:Loverich05}
    J. Loverich, A. Hakim, U. Shumlak,
    \newblock {\em A discontinuous Galerkin
    method for ideal two-fluid plasma equations,}
    \newblock Preprint submitted
    to Journal of Computational Physics, posted at
    \url{http://www.john-loverich.com/Loverich-JCP-2005.pdf},
    13 December 2005.

\end{thebibliography}

\end{document}
